\documentclass{article}
\usepackage{amsmath, amssymb}
\DeclareRobustCommand{\bbone}{\text{\usefont{U}{bbold}{m}{n}1}}
\DeclareMathOperator{\EX}{\mathbb{E}}% expected value

\begin{document}

\section{Define the GARCHX model}

\subsection{Mean Model}
\begin{align}
    r_t &= \mu + \sigma_t z_t \\
    z_t &\sim i.i.d\ N(0, 1)
\end{align}
where $e_t = \sigma_t z_t$.

\subsection{Volatility Model}
We introduce an exogenous term $x_t^2$ to the traditional GARCH model. The squared time series ensures non-negativity of $\sigma_t^2$ at any time step.
\begin{align}
    \sigma_t^2 &= \omega + \alpha e_{t-1}^2 + \beta \sigma_{t-1}^2 + \gamma x_t^2
\end{align}


\section{Log Likelihood}
For simplicity, assume $\mu=0$. Let $\theta = (\omega, \alpha, \beta, \gamma)$. Then the log likelihood function is defined as a function of the conditional densities of $r_t$ as such: 
\begin{align}
    l(\theta) &=  \sum_{t=1}^T \frac{1}{2} \left(-\log{2\pi} -\log{\sigma_t^2} - \frac{e_t^2}{\sigma_t^2} \right)
\end{align}

The first and second partial derivatives of $l(\theta)$ are as follows:
\begin{align}
    \frac{\partial }{\partial \theta} l(\theta) &=  \sum_{t=1}^T \frac{1}{2} \left(\frac{e_t^2}{\sigma_t^4} - \frac{1}{\sigma_t^2}\right) \frac{\partial \sigma_t^2}{\partial \theta}
\end{align}

\begin{align}
    \frac{\partial^2 }{\partial \theta_1 \partial \theta_2} l(\theta) &= - \frac{1}{2} \sum_{t=1}^T \left( \frac{\partial^2 \sigma_t^2}{\partial \theta_1 \partial \theta_2} \left(\frac{1}{\sigma_t^2} - \frac{e_t^2}{\sigma_t^4}\right) + \frac{\partial \sigma_t^2}{\partial \theta_1} \frac{\partial \sigma_t^2}{\partial \theta_2} \left(\frac{2e_t^2}{\sigma_t^6} - \frac{1}{\sigma_t^4}\right) \right) 
\end{align}

\section{Partial Derivatives of $\sigma^2$}
This section merely serves to supplement the previous section, as the first and second partial derivatives of $\sigma^2$ are used in the score funtion and information matrix. 

\subsection{First Partial Derivative}
\begin{align}
    \frac{\partial \sigma_{t}^2}{\partial \omega} &= 1 + \beta \frac{\partial \sigma_{t-1}^2}{\partial \omega} \\
    \frac{\partial \sigma_{t}^2}{\partial \alpha} &= e_{t-1}^2 + \beta \frac{\partial \sigma_{t-1}^2}{\partial \alpha} \\
    \frac{\partial \sigma_{t}^2}{\partial \beta} &= \sigma_{t-1}^2 + \beta \frac{\partial \sigma_{t-1}^2}{\partial \beta} \\
    \frac{\partial \sigma_{t}^2}{\partial \gamma} &= x_{t-1}^2 + \beta \frac{\partial \sigma_{t-1}^2}{\partial \gamma}
\end{align}

\subsection{Unconditional Expectation of First Derivatives of $\sigma^2$}
We first derive the unconditional expectation of $\sigma^2$ with respect to $\alpha$. The same logic is then applied to the other parameters. We have shown that $\EX(e_{t}^2)$ = $\EX(\sigma_{t}^2) = \frac{\omega + \gamma k}{1 - (\alpha+\beta)}$ for all $t$, where $k =\EX(x^2) $. Let $ \frac{\omega + \gamma k}{1 - (\alpha+\beta)} = C $ for brevity.
Then when $\beta < 1$, 

\[
\begin{split}
    \frac{\partial \sigma_{t}^2}{\partial \alpha} &= e_{t-1}^2 + \beta \frac{\partial \sigma_{t-1}^2}{\partial \alpha} \\
    \EX\left[\frac{\partial \sigma_{t}^2}{\partial \alpha}\right] &= C + \beta \EX\left[\frac{\partial \sigma_{t-1}^2}{\partial \alpha}\right]\\
     &= C + \beta \left( C + \beta \EX\left[\frac{\partial \sigma_{t-2}^2}{\partial \alpha}\right] \right) \\
     &= C + \beta C + \beta^2 \left( C + \beta \EX\left[\frac{\partial \sigma_{t-3}^2}{\partial \alpha}\right] \right) \\
     &= \sum_{i=0}^{\infty} C\beta^{i}\\
     &= \frac{C}{1-\beta}\\ 
     &= \frac{\omega + \gamma k}{(1-\beta)(1-(\alpha+\beta))}
\end{split}
\]

As long as $\beta < 1$, the unconditional expectation of first derivatives exist. Using the same logic, the first partial derivatives are as follows:

\begin{align}
    E\left[\frac{\partial \sigma_{t}^2}{\partial \omega}\right] &= \frac{1}{1-\beta} \\
    E\left[\frac{\partial \sigma_{t}^2}{\partial \alpha}\right] &= \frac{\omega + \gamma k}{(1-\beta)(1-\alpha-\beta)} \\
    E\left[\frac{\partial \sigma_{t}^2}{\partial \beta}\right] &= \frac{\omega + \gamma k}{(1-\beta)(1-\alpha-\beta)} \\
    E\left[\frac{\partial \sigma_{t}^2}{\partial \gamma}\right] &= \frac{k}{1-\beta}
\end{align}

\subsection{Second Derivative to $\sigma^2$}
There exist two cases. 

\textbf{Case a:} If both parameters $\theta_i, \theta_j$ are not $\beta$
\begin{align}
    \frac{\partial^2 \sigma_{t}^2}{\partial \theta_i \theta_j} &= \beta \frac{\partial^2 \sigma_{t-1}^2}{\partial \theta_i \theta_j}
\end{align}
with expectation:
\begin{align}
    E\left[\frac{\partial^2 \sigma_{t}^2}{\partial\theta_i \partial\theta_j}\right] = 0
\end{align}

\textbf{Case b:} If at least one of the parameters $\theta_i, \theta_j$ is $\beta$
\begin{align}
    \frac{\partial^2 \sigma_{t}^2}{\partial \beta \theta_j} &= \frac{\partial \sigma_{t-1}^2}{\partial \theta_j } + \beta \frac{\partial^2 \sigma_{t-1}^2}{\partial \beta \theta_j}
\end{align}


\section{Stationarity of Time Series}

\subsection{Stationarity of $\sigma_t^2$}

We know that $e_t = \sigma_t z_t$, since $z_t \sim i.i.d \ N(0,1)$,

\[
\begin{split}
E[e_t^2] &= E[\sigma_t^2 z_t^2] \\
&= E[\sigma_t^2] E[z_t^2] \\
&= E[\sigma_t^2]
\end{split}
\]

\textit{Note that $E[z_t^2]$ is the expectation of a chi-squared distribution with 1 degree of freedom.}

Using the fact that $E[e_t^2] = E[\sigma_t^2]$,
\[
\begin{split}
    E[\sigma_t^2] &= E[\omega + \alpha \sigma_{t-1}^2 + \beta e_{t-1}^2 + \gamma x_t^2] \\
    &= E[\omega] + \alpha E[\sigma_{t-1}^2] + \beta E[e_{t-1}^2] + \gamma E[x_t^2] \\
    &= \omega + \gamma E[x_t^2] + \alpha E[\sigma_{t-1}^2] + \beta E[e_{t-1}^2] \\
    &= \omega + \gamma k + (\alpha + \beta) E[\sigma_{t-1}^2] \\
    &= \omega + \gamma k + (\alpha + \beta) E[\omega + \alpha \sigma_{t-2}^2 + \beta e_{t-2}^2 + \gamma x_{t-1}^2] \\
    &= \ldots \\
    &= (\omega + \gamma k) + [1+ (\alpha + \beta) + (\alpha + \beta)^2 + (\alpha + \beta)^3 + \ldots] \\
    &= \frac{\omega + \gamma k}{1-(\alpha+\beta)}
\end{split}
\]

where $k = E[x_t^2]$, which is a constant, given that we assume $x_t^2$ is stationary.

From this, we know that $|\alpha + \beta| < 1$.

Since $x_t^2$ is stationary, then $E[x_t^2, x_{t-s}^2]$ is not dependent on $t$.

Following that $x_t^2$ is stationary, we prove that

\[
\begin{split}
Cov[\sigma_t^2,\sigma_{t-s}^2] &= E[\sigma_t^2 \sigma_{t-s}^2] - E[\sigma_t^2]E[\sigma_{t-s}^2] \\
&= E[\sigma_t^2 \sigma_{t-s}^2] - \left(\frac{\omega + \gamma k}{1-(\alpha+\beta)}\right)^2
\end{split}
\]

We found 

\[
\begin{split}
E[\sigma_t^2 \sigma_{t-s}^2] &= E[(\omega + \alpha \sigma_{t-1}^2 + \beta e_{t-1}^2 + \gamma x_t^2)(\omega + \alpha \sigma_{t-2}^2 + \beta e_{t-2}^2 + \gamma x_{t-1}^2)] \\
&= \ldots \\
&= (\omega^2 + 2\gamma \omega k) + (2 \alpha \omega + 2 \beta \omega + 2 \alpha \gamma k + 2 \beta \gamma k) \left(\frac{\omega + \gamma k}{1-(\alpha+\beta)}\right) \\
&\quad + (\alpha + \beta)^2 E[\sigma_{t-1}^2 \sigma_{t-s-1}^2] + \gamma^2 E[x_t^2 x_{t-s}^2]
\end{split}
\]

Since we know $x_t$ is stationary, $Cov[x_t^2,x_{t-s}^2] = E[x_t^2 x_{t-s}^2] - E[x_t^2]E[x_{t-s}^2]$ does not depend on $t$, thus we can conclude that $E[x_t^2 x_{t-s}^2]$ does not depend on $t$.

This shows that $Cov[\sigma_t^2,\sigma_{t-s}^2]$ does not depend on $t$.

We see that the following is a stationary AR(1) process now as $(\alpha+\beta) < 1 \implies  (\alpha+\beta)^2 < 1$:

\[
\begin{split}
&E[\sigma_t^2 \sigma_{t-s}^2] = (\omega^2 + 2\gamma \omega k) + 2(\alpha\omega + \beta\omega + \alpha\gamma k + \beta\gamma k) \frac{\omega+\gamma k}{1-(\alpha+\beta)} \\
&\quad + (\alpha+\beta)^2 E[\sigma_{t-1}^2\sigma_{t-s-1}^2] + \gamma^2 E[x_t^2 x_{t-s}^2] \\
&\implies E[\sigma_t^2 \sigma_{t-s}^2] \text{ does not depend on }t \\
&\implies cov(\sigma_t^2, \sigma_{t-s}^2) = E[\sigma_t^2 \sigma_{t-s}^2] - \left(\frac{\omega+\gamma k}{1-(\alpha+\beta)}\right)^2 \text{ does not depend on }t \\
&\therefore \{\sigma_t^2\} \text{ is stationary}.
\end{split}
\]

\subsection{Stationarity of $r_t$}

First, note that $z_t \sim i.i.d \ N(0,1) \implies E[z_t] = 0$.

To show $E[r_t]$ is constant:

\[
\begin{split}
E[r_t] &= E[\mu + \sigma_t z_t] \\
        &= \mu + E[\sigma_t z_t] \\
        &= \mu + E[\sigma_t]E[z_t] \\
        &= \mu
\end{split}
\]

Next, we show finite variance of $r_t$:

\[
\begin{split}
Var(r_t) &= E[\sigma_t^2] \\ 
        &= \frac{\omega + \gamma k}{1-(\alpha+\beta)}
\end{split}
\]

Lastly, we show that the autocovariance of $r_t$ is not time dependent:

\[
\begin{split}
Cov(r_t, r_{t-r}) &= Cov(e_t, e_{t-r}) \\ 
        &= E[e_t e_{t-r}] -E[e_t] E[e_{t-r}] \\ 
        &= 0
\end{split}
\]

\subsection{Convergence of $e_t$}

We know that $e_t = \sigma_t z_t$,

and $z_t$ is

\[
z_t  \sim i.i.d \ N(0, 1)
\]

Therefore,

\[
\begin{split}
    E[e_{t}| \mathcal{F}_{t-1}] &= E[\sigma_t z_t| \mathcal{F}_{t-1}] \\
    &= E[z_{t}| \mathcal{F}_{t-1}] \ E[\sigma_t| \mathcal{F}_{t-1}] \\
    &= 0
\end{split}
\]

Thus, we conclude that $e_t$ is an MDS sequence.

For simplicity, we assume our time series is ergodic.

By the Martingale CLT, as $e_t$ is a stationary and ergodic Martingale difference,

\[
\frac{1}{\sqrt{n}} \sum_{t=1}^n e_t \xrightarrow[]{\text{d}} N(0, \sigma^2)
\]


\section{Consistency of the MLE}

Let the score function be defined as $s(\theta) = \frac{\partial}{\partial\theta} l(\theta)$ and the information $I(\theta) = \frac{\partial^2}{\partial\theta^2} l(\theta)$. 

Under suitable regularity conditions, and having shown that the time series is stationary and ergodic,

\[
\hat{\theta} \xrightarrow[]{\text{p}} \theta_0
\]


\section{Asymptotic normality of the MLE}

First, we consider the distribution of the true score $s(\theta_0)$. Under the regularity conditions, it can be shown that $E(s_n(\theta_0)) = 0$ and $Var(s_n(\theta_0)) = I_E(\theta_0)$. Furthermore, as we have shown that $r_t$ converges in distribution by the Martingale CLT, the score function

\[
\begin{split}
 \frac{1}{\sqrt{n}} s_n(\theta_0) &= \sum_{t=1}^n \frac{\partial}{\partial\theta}\log f(r_t | \theta_0) \\ 
 &\xrightarrow[]{\text{d}} N(0, I_E(\theta_0))
\end{split}
\]

converges in distribution to a multivariate normal distribution $N(0, I_E(\theta_0))$.

Next, consider a first-order multivariate Taylor expansion of the score function at $\theta_0$ about $\hat{\theta}$:

\[
s_n(\theta_0) \approx s_n(\hat{\theta}) - I_n(\theta_0) (\theta_0 - \hat{\theta})
\]

Since by definition $s_n(\hat{\theta}) = 0$,

\[
\begin{split}
 s_n(\theta_0) &= - I_n(\theta_0) (\theta_0 - \hat{\theta}) \\
 (\hat{\theta} - \theta_0) &=  I_n(\theta_0)^{-1} s_n(\theta_0) \\
 \sqrt{n}(\hat{\theta} - \theta_0) &=  \sqrt{n} I_n(\theta_0)^{-1} s_n(\theta_0) 
\end{split}
\]

As shown earlier,

\[
\frac{1}{\sqrt{n}} s_n(\theta_0)  \xrightarrow[]{\text{d}} N(0, I_E(\theta_0))
\]

and by the law of large numbers,

\[
\frac{1}{n} I_n(\theta_0) = \frac{1}{n} \sum_{t=1}^n \frac{\partial^2}{\partial\theta^2} \log(f(x_i|\theta))  \xrightarrow[]{\text{p}} I_E(\theta_0)
\]

Therefore, by Slutsky's theorem,

\[
\sqrt{n}(\hat{\theta} - \theta_0) \xrightarrow[]{\text{d}} I_E(\theta_0)^{-1} N(0, I_E(\theta_0)) = N(0, I_E(\theta_0)^{-1})
\]




\end{document}